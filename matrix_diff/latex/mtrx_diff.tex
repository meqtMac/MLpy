\documentclass{article}
\usepackage{amsmath}

\title{Matrix Calculus Note}
\author{Yi Jiang}

\begin{document}
    \maketitle
    \newtheorem{theorem}{Theorem}

\section{Introductory Example}
\begin{equation}
    \frac{\partial f}{\partial x} = a
\end{equation}
for multivariate, we have:
$$ f(x) = \sum_i a_i x_i = a^T x $$
$$ \frac{\partial f }{\partial x_k} = \frac{\partial(\sum_i a_i x_i)}{\partial x_k } = a_k $$
Thenn we organize n partial derivatives in the following way:
\begin{equation}
    \frac{\partial f }{\partial x } = 
    \begin{bmatrix}
        \frac{\partial f}{\partial x_1 } \\
        \frac{\partial f}{\partial x_2} \\
        \vdots \\
        \frac{\partial f}{\partial x_n} 
    \end{bmatrix}
    = \begin{bmatrix}
        a_1 \\
        a_2 \\
        \vdots \\
        a_n
    \end{bmatrix}
    = a
\end{equation}


\section{Derivation}

\subsection{Organization of Elements}

For a scalar valued function $f(x)$, the result of $\frac{\partial f}{\partial x} $ has the same size with x. That 
\begin{equation}
    \frac{\partial f}{\partial x} =
    \begin{bmatrix}
        \frac{\partial f}{\partial x_{11}} & \frac{\partial f}{\partial x_{12}} & \cdots & \frac{\partial f}{\partial x_{1n}} \\
        \frac{\partial f}{\partial x_{21}} & \frac{\partial f}{\partial x_{22}} & \cdots & \frac{\partial f}{\partial x_{2n}} \\
        \vdots & \vdots & \ddots & \vdots \\
        \frac{\partial f}{\partial x_{m1}} & \frac{\partial f}{\partial x_{m2}} & \cdots & \frac{\partial f}{\partial x_{mn}}
    \end{bmatrix}
\end{equation}
By this definition, we have:
$$ \frac{\partial f}{\partial x} = ( \frac{\partial f}{\partial x})^T = a^T $$

\subsection{Deal with Inner Product}
For $f(x) = a^T x$, we have$ \frac{\partial f}{\partial
x} = a $.

\subsection{Properties of Trace}
Defintion 2. Trace $ Tr [A] = \sum_i A_{ii} $
$\frac{\partial Tr [A] }{\partial A} = I$
\begin{theorem}
    Matrix traces has the following properties.
    \begin{enumerate}
        \item Tr[$A^TB$] = $\sum_i\sum_j A_{ij}B{ij}$
    \end{enumerate}
\end{theorem}
If therel's a multivariate scalar function $f(x)$ = Tr[$A^Tx$], we have $\frac{\partial f}{\partial x} = A $.

\subsection{Deal with Generalized Inner Product}

\subsection{Define Matrix Differential}
dTr[$A$] = Tr[d$A$]

d$f$ = Tr$\left[ (\frac{\partial f}{\partial x})^T d x \right] $

\subsection{Matrix Differential Properties}

eg. 4. Given function $ f(x) = x^T A x$, where $A$ is square and x is a column vertor, we can compute:

eg. 5. $d(X^{-1}) = -X^{-1}dX X^{-1} $.

\subsection{Schame of Handlding Scalar Function}

\subsection{Determinant}





\end{document}